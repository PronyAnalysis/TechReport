\section{Introduction}

Consider the following system of signals:

$$
  S = \{ (a_j, x_j) \}, \quad j = 1, \ldots , N
$$

where $a_j$ are signal amplitudes and $x_j$ are signal positions. $\{a_j\}$ and $\{x_j\}$ may be represent by real or complex numbers. \par

Define the Prony map as a map $\{ (a_j, x_j) \} \mapsto \{m_k\}$ of the following form:
$$
  \label{eq:prony}
  m_k = \sum_{j=1}^N a_j x_j^k.
$$
  
Numbers $\{m_k\}$ are moments of the system, equation \eqref{eq:prony} is the Prony equation system.

There is a special case of the Prony system:

$$
  S_t = \{ (a_j, \mu_j) \}; \quad m_k = \sum_{j=1}^N a_j e^{-2\pi i \mu_j k \delta}
$$

where $\delta \in \mathbb{R}_+$ is a parameter. We say that it is the trigonometric Prony map. If $|x_j| = 1$, we can reduce general Prony map to the trigonometric case by $x_j = e^{-2\pi i \mu_j \delta}$.

In the practical application, we usually know only moments of a source system and we want to reconstruct original amplitudes and positions of signals. This means that we should build the inverse Prony map. Unfortunately, measurements of moments frequently include some noise. A feature of the inverse Prony map is the following: even small noise can produce a significant reconstruction error. Therefore, it is very important to understand geometry of the Prony map and the inverse Prony map.

In this technical report, we consider some examples of the direct and inverse Prony map and build corresponding plots. You can also view the source code on the R language for each plot.

See also: \cite{2015arXiv150206932A}, \cite{azais_spike}, \cite{batenkov_numerical_2014}, \cite{batenkov_accurate_2014}, \cite{Bat.Sar.Yom}, \cite{Bat.Yom2}, \cite{Bat.Yom.Sampta13}, \cite{Bat.Yom1}, \cite{candes_towards_2014}, \cite{candes_super-resolution_2013}, \cite{demanet_super-resolution_2013}, \cite{donoho_superresolution_1992}, \cite{Don1}, \cite{duval_exact_2013}, \cite{fernandez-granda_support_2013}, \cite{heckel_super-resolution_2014}, \cite{Lev.Ful}, \cite{liao_music_2014}, \cite{McC}, \cite{Min.Kaw.Min}, \cite{moitra_threshold_2014}, \cite{Ode.Bar.Pis}, \cite{Sle}, \cite{Yom2}, \cite{Yom1}, \cite{Dem.Ngu}, \cite{Mor.Can}.