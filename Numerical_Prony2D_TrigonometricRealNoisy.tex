\subsection{Trigonometric noisy real Prony, 2D}

General solving approach for general complex Prony problems doesn't work well for noisy system. Consider the following numerical example: we have two-dimensional trigonometry Prony system ($S_1$). Let's add some noise to the moments and try to reconstruct original system by the general approach ($S_2$) and the trigonometric approach ($S_3$).
The numerical results are in the table \ref{tbl:noisy}.

\begin{table}
\label{tbl:noisy}
\centering
\begin{tabular}{rllllllll}
  \hline
 & $x_1$ & $x_2$ & $a_1$ & $a_2$ & $m_0$ & $m_1$ & $m_2$ & $m_3$ \\ 
  \hline
  $S_1$ & 0.891-0.454i & 0.951-0.309i & 1+0i & 1+0i & 2+0i & 1.84-0.76i & 1.4-1.4i & 0.74-1.8i \\ 
  $S_2$ & 0.919-0.385i & 1.93-6.15i & 2.04+0.09i & 0.0005-0.0002i & 2.04+0.09i & 1.91-0.71i & 1.46-1.39i & 0.76-1.7i \\ 
  $S_3$ & 0.953-0.304i & 0.916-0.402i & 1.02+0i & 1.02+0i & 2.04+0.09i & 1.91-0.71i & 1.46-1.39i & 0.76-1.7i \\ 
  $S_2-S_1$ & 0.0277+0.0691i & 0.98-5.84i & 1.04+0.09i & -0.999-0i & 0.0425+0.0877i & 0.067+0.0542i & 0.0624+0.0105i & 0.019+0.0951i \\ 
  $S_3-S_1$ & 0.062+0.15i & -0.0355-0.0932i & 0.0222+0i & 0.0222+0i & 0.0425+0.0877i & 0.067+0.0542i & 0.0624+0.0105i & 0.019+0.0951i \\ 
   \hline
\end{tabular}
\end{table}

\midskip
  
We can see that the general approach gives a big error for $x_2$ and $a_2$. The trigonometric approach solves the noisy problem well.
  